\subsection{Giới thiệu về công cụ Selenium}

Selenium là một bộ công cụ mã nguồn mở hỗ trợ kiểm thử tự động các ứng dụng web. Ra đời vào năm 2004, Selenium đã trở thành một trong những công cụ phổ biến nhất cho việc kiểm thử giao diện người dùng (UI) nhờ tính linh hoạt, khả năng hỗ trợ đa nền tảng và ngôn ngữ lập trình.

\begin{figure}[H]
    \centering
    \includegraphics[width=0.5\linewidth]{image/selenium.png}
    \caption{Logo của Selenium}
\end{figure}

Selenium bao gồm nhiều thành phần, nhưng các thành phần chính thường được sử dụng bao gồm:
\begin{itemize}
    \item \textbf{Selenium WebDriver}: Công cụ mạnh mẽ nhất trong bộ Selenium, cho phép thực hiện kiểm thử tự động bằng cách điều khiển trình duyệt một cách trực tiếp.
    \item \textbf{Selenium IDE}: Một tiện ích mở rộng của trình duyệt cho phép ghi lại các kịch bản kiểm thử đơn giản và phát lại chúng.
    \item \textbf{Selenium Grid}: Hỗ trợ chạy kiểm thử song song trên nhiều máy và trình duyệt, tối ưu hóa thời gian kiểm thử.
\end{itemize}

Các tính năng chính của Selenium:
\begin{itemize}
    \item Hỗ trợ đa trình duyệt: Chrome, Firefox, Safari, Edge, v.v.
    \item Tương thích với nhiều ngôn ngữ lập trình: Java, Python, C\#, Ruby, JavaScript.
    \item Được tích hợp tốt với các công cụ kiểm thử và CI/CD như Jenkins, Maven.
\end{itemize}

\subsection{Lý do sử dụng Selenium}

Selenium được lựa chọn làm công cụ kiểm thử tự động trong nhiều dự án vì những lý do sau:
\begin{enumerate}
    \item \textbf{Mã nguồn mở và miễn phí}: Selenium không yêu cầu bất kỳ chi phí bản quyền nào, phù hợp cho cả dự án cá nhân và doanh nghiệp.
    \item \textbf{Hỗ trợ đa nền tảng}: Với Selenium WebDriver, bạn có thể kiểm thử trên nhiều hệ điều hành (Windows, macOS, Linux) và trình duyệt (Chrome, Firefox, Edge, v.v.), đảm bảo tính tương thích của ứng dụng trên các môi trường khác nhau.
    \item \textbf{Khả năng mở rộng cao}: Selenium cung cấp một API mạnh mẽ và linh hoạt, cho phép tích hợp với các framework kiểm thử như TestNG, JUnit (Java), Pytest (Python), hoặc các công cụ như Jenkins, Docker.
    \item \textbf{Hỗ trợ kiểm thử nâng cao}: Selenium không chỉ kiểm thử giao diện mà còn cho phép tương tác với các yếu tố phức tạp trên trang web như các iframe, pop-up, hoặc Ajax.
    \item \textbf{Cộng đồng phát triển lớn}: Với một cộng đồng người dùng rộng lớn, tài liệu phong phú và các diễn đàn hỗ trợ, Selenium giúp người mới dễ dàng học tập và giải quyết các vấn đề phát sinh.
\end{enumerate}

Selenium phù hợp với các dự án yêu cầu kiểm thử tự động ở quy mô lớn, đặc biệt là các ứng dụng web cần đảm bảo hoạt động ổn định trên nhiều môi trường khác nhau.

\subsection{Cấu trúc bài báo cáo}
Bài báo cáo bao gồm bốn phần chính:
\begin{itemize}
    \item \textbf{Phần 1: Giới thiệu}: Trình bày về Selenium, các thành phần và tính năng chính; lý do chọn Selenium làm công cụ kiểm thử tự động; Cấu trúc bài báo cáo; Mô tả môi trường kiểm thử.
    \item \textbf{Phần 2: Mô tả}: Trình bày về các chức năng cần kiểm thử, mô tả cách hiện thực chương trình kiểm thử tự động  và dữ liệu input của chương trình ở cả 2 levels.
    \item \textbf{Phần 3: Kết quả}: Trình bày kết quả kiểm thử của chương trình.
\end{itemize}

\subsection{Môi trường và cách thiết lập kiểm thử}
\subsubsection{Môi trường kiểm thử}
Môi trường kiểm thử bao gồm:
\begin{itemize}
    \item \textbf{Hệ điều hành}: Windows 10 và macOS Sonoma 14.7
    \item \textbf{Trình duyệt}: Google Chrome 131.0.6778.109
    \item \textbf{Ngôn ngữ lập trình}: Python 3.13.0
    \item \textbf{Công cụ kiểm thử}: Selenium 4.26.1
\end{itemize}

\subsubsection{Cách thiết lập kiểm thử}

Các bước thiết lập kiểm thử:
\begin{enumerate}
    \item Install packages: \texttt{pip install selenium pytest}
    \item Chạy lệnh pytest với từng tính năng và level, ví dụ với tính năng Create Quiz level 1, ta chạy lệnh: \texttt{pytest level1/CreateQuiz}
\end{enumerate}