\subsection{Private Message}
\subsubsection{Level 1}
Ở level 1, mỗi testcase bao gồm các trường:
\begin{itemize}
    \item \textbf{id}: Mã số của testcase
    \item \textbf{message}: Nội dung tin nhắn được gửi đi.
    \item \textbf{expected\_text}: Nội dung tin nhắn kết quả để kiếm tra so sánh trùng khớp.
\end{itemize}

\noindent Ví dụ:
\begin{lstlisting}[language=bash, caption={Ví dụ testcase PM-001-0001 ở level 1}, breaklines=true]
    {
      "id": "PM-001-001",
      "message": "ABCDEFMLAKSPQKAMLAMN",
      "expected_text": "ABCDEFMLAKSPQKAMLAMN"
    }
\end{lstlisting}

\subsubsection{Level 2}
Ở level 2, các trường chung bao gồm:

\begin{itemize}
    \item \textbf{url}: Đường dẫn trang web.
    \item \textbf{username}: Tên đăng nhập.
    \item \textbf{password}: Mật khẩu.
\end{itemize}
Tiếp theo là các trường \texttt{selectors} với các giá trị bên trong bao gồm \texttt{type} (ví dụ: ID, XPATH, CSS\_SELECTOR,..) và \texttt{value}. Các \texttt{selectors} ở đây gồm có:
\begin{itemize}
    \item \textbf{login\_link}: Liên kết đăng nhập.
    \item \textbf{username\_input}: Trường nhập tên đăng nhập.
    \item \textbf{password\_input}: Trường nhập mật khẩu.
    \item \textbf{login\_button}: Nút đăng nhập.
    \item \textbf{messaging\_drawer\_toggle}: Nút bật/tắt bảng tin nhắn.
    \item \textbf{messages\_overview\_toggle}: Nút chuyển đổi xem tổng quan tin nhắn.
    \item \textbf{contact\_select}: Lựa chọn liên hệ.
    \item \textbf{message\_input}: Trường nhập tin nhắn.
    \item \textbf{send\_button}: Nút gửi tin nhắn.
    \item \textbf{message\_container}: Vùng chứa tin nhắn.
    \item \textbf{message\_text}: Nội dung tin nhắn.
    \item \textbf{window\_size}: Kích thước cửa sổ.
\end{itemize}

\noindent Tiếp đó là trường \texttt{testcase}, mỗi testcase sẽ có các trường:
\begin{itemize}
    \item \textbf{id}: Mã số của testcase
    \item \textbf{message}: Nội dung tin nhắn được gửi đi.
    \item \textbf{expected\_text}: Nội dung tin nhắn kết quả để kiếm tra so sánh trùng khớp.
\end{itemize}

\noindent Ví dụ:
\begin{lstlisting}[language=bash, breaklines=true]
    "url": "https://school.moodledemo.net/",
    "login_credentials": {
        "username": "amandahamilto205",
        "password": "moodle"
    },
    "selectors": {
        "login_link": {
          "type": "LINK_TEXT",
          "value": "Log in"
        },
        "username_input": {
          "type": "ID",
          "value": "username"
        },
    },
    ...
    "test_cases": [
    {
        "id": "PM-001-001",
        "message": "ABCDEFMLAKSPQKAMLAMN",
        "expected_text": "ABCDEFMLAKSPQKAMLAMN"
    },
\end{lstlisting}

\subsection{Private File Upload}
\subsubsection{Level 1}
Ở level 1, mỗi testcase bao gồm các trường:
\begin{itemize}
    \item \textbf{id}: Mã số của testcase
    \item \textbf{file\_path}: Đường dẫn file từ máy tính. Lấy file từ: \href{https://drive.google.com/drive/folders/1mAcoKv5-AXE3Vms1LvrvOoVR0mKdqAaT?usp=sharing}{Link file test}.
    \item \textbf{expected\_message}: Kết quả thông báo mong đợi sau khi upload file.
\end{itemize}
\noindent Ví dụ:
\begin{lstlisting}[language=bash, breaklines=true]
    {
      "id": "PF-001-002",
      "file_path": "D:\\1byte.txt",
      "expected_message": "Changes saved"
    }
\end{lstlisting}

\subsubsection{Level 2}
Ở level 2, các testcase sẽ có các trường chung như sau:
\begin{itemize}
    \item \textbf{url}: Đường dẫn trang web.
    \item \textbf{window\_size}: Kích thước cửa sổ (bao gồm chiều rộng và chiều cao).
    \item \textbf{credentials}: Thông tin đăng nhập (bao gồm tên người dùng và mật khẩu).
    \item \textbf{selectors}: Các bộ chọn cho các phần tử trên trang web.
    \begin{itemize}
        \item \textbf{login\_link}: Liên kết đăng nhập.
        \item \textbf{username\_field}: Trường nhập tên đăng nhập.
        \item \textbf{password\_field}: Trường nhập mật khẩu.
        \item \textbf{login\_button}: Nút đăng nhập.
        \item \textbf{user\_menu}: Menu người dùng.
        \item \textbf{private\_files}: Tệp riêng tư.
        \item \textbf{add\_file\_button}: Nút thêm tệp.
        \item \textbf{file\_input}: Trường nhập tệp.
        \item \textbf{upload\_button}: Nút tải lên.
        \item \textbf{error\_message}: Thông báo lỗi.
        \item \textbf{submit\_button}: Nút gửi.
        \item \textbf{toast\_message}: Thông báo kiểu toast.
    \end{itemize}
    \item \textbf{timeouts}: Thời gian chờ cho các thao tác.
    \begin{itemize}
        \item \textbf{error\_wait}: Thời gian chờ cho thông báo lỗi.
        \item \textbf{element\_wait}: Thời gian chờ cho phần tử xuất hiện.
        \item \textbf{small\_file\_wait}: Thời gian chờ cho tệp nhỏ.
        \item \textbf{large\_file\_wait}: Thời gian chờ cho tệp lớn.
    \end{itemize}
    \item \textbf{file\_size\_threshold}: Ngưỡng kích thước tệp.
\end{itemize}

\noindent Ví dụ:
\begin{lstlisting}[language=bash, breaklines=true]
      "url": "https://school.moodledemo.net/",
      "window_size": {
        "width": 1296,
        "height": 696
      },
      "credentials": {
        "username": "amandahamilto205",
        "password": "moodle"
      },
      "selectors": {
        "login_link": "Log in",
        "username_field": "username",
        "password_field": "password",
        "login_button": "loginbtn"
        ....
      },
      "timeouts": {
        "error_wait": 3,
        "element_wait": 10,
        "small_file_wait": 15,
        "large_file_wait": 2700
      },
      "file_size_threshold": 90000000,
\end{lstlisting}
\vspace{10pt}
\noindent Tiếp theo đó là tập hợp các testcase như level 1.