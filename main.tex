\documentclass[a4paper, 12pt]{article}
\usepackage{vntex}
\usepackage{a4wide,amssymb,epsfig,latexsym,array,hhline,fancyhdr}
\usepackage[normalem]{ulem}
\usepackage[makeroom]{cancel}
\usepackage{amsmath}
\usepackage{amsthm}
\usepackage{multicol,longtable,amscd}
\usepackage{diagbox}%Make diagonal lines in tables
\usepackage{booktabs}
\usepackage{alltt}
\usepackage[framemethod=tikz]{mdframed}% For highlighting paragraph backgrounds
\usepackage{caption,subcaption}
\usepackage{lastpage}
\usepackage[lined,boxed,commentsnumbered]{algorithm2e}
\usepackage{enumerate}
\usepackage{color}
\usepackage{xcolor}
\usepackage{graphicx}							% Standard graphics package
\usepackage{array}
\usepackage{tabularx, caption}
\usepackage{multirow}
\usepackage{multicol}
\usepackage{rotating}
\usepackage{graphics}
\usepackage{geometry}
\usepackage{setspace}
\usepackage{epsfig}
\usepackage{tikz}
\usepackage{listings}
\usepackage{float}
\usetikzlibrary{arrows,snakes,backgrounds}
\usepackage{hyperref}
\hypersetup{urlcolor=black,linkcolor=black,citecolor=black,colorlinks=true} 
%\usepackage{pstcol} 								% PSTricks with the standard color package

\newtheorem{theorem}{{\bf Theorem}}
\newtheorem{property}{{\bf Property}}
\newtheorem{proposition}{{\bf Proposition}}
\newtheorem{corollary}[proposition]{{\bf Corollary}}
\newtheorem{lemma}[proposition]{{\bf Lemma}}

\AtBeginDocument{\renewcommand*\contentsname{Mục lục}}
\AtBeginDocument{\renewcommand*\refname{Tài liệu tham khảo}}
\AtBeginDocument{\renewcommand*\listfigurename{Tiêu đề ảnh}}
%\usepackage{fancyhdr}
\setlength{\headheight}{40pt}
\pagestyle{fancy}
\fancyhead{} % clear all header fields
\fancyhead[L]{
 \begin{tabular}{rl}
    \begin{picture}(25,15)(0,0)
    \put(0,-8){\includegraphics[width=8mm, height=8mm]{hcmut.png}}
   \end{picture}&
	\begin{tabular}{l}
		\textbf{\bf \ttfamily Trường Đại Học Bách Khoa Tp.Hồ Chí Minh}\\
		\textbf{\bf \ttfamily Khoa Khoa Học \& Kỹ Thuật Máy Tính}
	\end{tabular} 	
 \end{tabular}
}
\fancyhead[R]{
	\begin{tabular}{l}
		\tiny \bf \\
		\tiny \bf 
	\end{tabular}  }
\fancyfoot{} % clear all footer fields
\fancyfoot[L]{\scriptsize \ttfamily Báo cáo môn Kiểm tra phần mềm - Học kì 241}
\fancyfoot[R]{\scriptsize \ttfamily Trang {\thepage}/\pageref{LastPage}}
\renewcommand{\headrulewidth}{0.3pt}
\renewcommand{\footrulewidth}{0.3pt}


%%%
\setcounter{secnumdepth}{4}
\setcounter{tocdepth}{3}
\makeatletter
\newcounter {subsubsubsection}[subsubsection]
\renewcommand\thesubsubsubsection{\thesubsubsection .\@alph\c@subsubsubsection}
\newcommand\subsubsubsection{\@startsection{subsubsubsection}{4}{\z@}%
                                     {-3.25ex\@plus -1ex \@minus -.2ex}%
                                     {1.5ex \@plus .2ex}%
                                     {\normalfont\normalsize\bfseries}}
\newcommand*\l@subsubsubsection{\@dottedtocline{3}{10.0em}{4.1em}}
\newcommand*{\subsubsubsectionmark}[1]{}
%\renewcommand{\thesubsubsection}{\alph{subsubsection}}
\everymath{\color{blue}}
\makeatother

\begin{document}

\begin{titlepage}
\begin{center}
VIETNAM NATIONAL UNIVERSITY, HO CHI MINH CITY \\
UNIVERSITY OF TECHNOLOGY \\
FACULTY OF COMPUTER SCIENCE AND ENGINEERING
\end{center}
\vspace{1cm}

\begin{figure}[h!]
\begin{center}
\includegraphics[width=3cm]{hcmut.png}
\end{center}
\end{figure}

\vspace{1cm}


\begin{center}
\begin{tabular}{c}
\multicolumn{1}{l}{\textbf{{\Large KIỂM TRA PHẦN MỀM (CO3015)}}}\\

~~\\
\hline
\\
\\
\textbf{{\Huge BTL 2: BLACK BOX TESTING}}\\
\\

\\
\hline
\end{tabular}
\end{center}

\vspace{3cm}

\begin{table}[h]
\begin{tabular}{rrl}
\hspace{3 cm} & Giảng viên hướng dẫn: & Bùi Hoài Thắng\\

& Sinh viên thực hiện:
& Phạm Đức Hào - 2111128\\
&& Hồ Trọng Nhân - 2111899\\
&& Đậu Đức Quân - 2114531\\
&& Nguyễn Phúc Minh Quân - 2110479\\
&& Trần Mậu Thật - 2112342
\end{tabular}
\end{table}

\begin{center}
{\footnotesize HO CHI MINH CITY, NOVEMBER 2024}
\end{center}
\end{titlepage}


\newpage
\tableofcontents
\newpage
\section{Giới thiệu:}

\section{Phân công}
 \begin{table}[H]
\centering
\begin{tabular}{|p{3cm}|p{3cm}|l|c|}
\hline 
Reviewer &
Validator &
  \multicolumn{1}{c|}{Feature} &Contributon \\ \hline
Đậu Đức Quân & Trần Mậu Thật &&\\\hline
Hồ Trọng Nhân & Đậu Đức Quân &&\\ \hline
Nguyễn Phúc Minh Quân & Hồ Trọng Nhân &&\\ \hline
Phạm Đức Hào & Nguyễn Phúc Minh Quân &&\\ \hline
Trần Mậu Thật & Phạm Đức Hào &&\\ \hline
\end{tabular}
\caption{Bảng phân công công việc}
\label{tab:my-table}
\end{table}





\newpage
\section{Quá trình làm việc nhóm:}
\subsection{Buổi họp trao đổi những thắc mắc của các thành viên về bài tập lớn và phân công công việc. (15/11/2024)}
Nội dung chính:
\begin{itemize}
    \item
\end{itemize}
Kết quả cuộc họp:
\begin{itemize}
  \item
\end{itemize}

\newpage

\section{Kết quả kiểm thử}

\subsection{Tính năng 1}
\subsubsection{Mô tả}
\subsubsection{Test case}
\subsection{Tính năng 2}

\subsection{Chức năng Tạo sự kiện (Create event)}
\subsubsection{Use case}
\begin{table}[H]
    \centering
    \begin{tabular}{|l|p{11cm}|}
        \hline
        Category & Description \\
        \hline
        Use case name & Create event\\
        \hline
        Actor & Manager \\
        \hline
        Assumption & Người dùng ở trang \url{https://sandbox.moodledemo.net/}. Ngôn ngữ trang web được chỉnh tiếng Việt. Người dùng đăng nhập thành công dưới bất kì tài khoản nào. Người dùng đang ở trang "Bảng điều khiển". \\
        \hline
        Normal flow & 1. Người dùng nhấn vào nút "Sự kiện mới". \\
        & 2. Hệ thống hiển thị bảng "Sự kiện mới". \\
        & 3. Người dùng nhập thông tin có thể bao gồm chữ, số và kí tự vào ô "Tên sự kiện". \\
        & 4. Người dùng chọn thông tin ngày, tháng, năm, giờ, phút của sự kiện. \\
        & 5. Người dùng chọn "Thành viên" hoặc "Hệ thống" tại hàng "Loại sự kiện".\\
        & 6. Người dùng nhấn vào nút "Lưu". \\
        & 7. Hệ thống đóng bảng "Sự kiện mới". \\
        \hline
        Alternative flows & A1. Tại bước 6: \\
        & 6.1. Người dùng nhấn vào nút "Show more". \\
        & 6.2. Người dùng nhập thông tin vào ô "Mô tả". \\
        & Quay lại bước 6 trên Normal flow. \\
        & \\
        & A2. Tại bước 6: \\
        & 6.1. Người dùng nhấn vào nút "Show more". \\
        & 6.2. Người dùng nhập thông tin vào ô "Địa chỉ". \\
        & Quay lại bước 6 trên Normal flow. \\
        & \\
        & A3. Tại bước 6: \\
        & 6.1. Người dùng nhấn vào nút "Show more". \\
        & 6.2. Người dùng chọn lựa chọn "Tới" tại hàng "Thời lượng". \\
        & 6.3. Người dùng chọn thông tin ngày, tháng, năm, giờ, phút kết thúc của sự kiện. \\
        & Quay lại bước 6 trên Normal flow. \\
        & \\
        & A4. Tại bước 6: \\
        & 6.1. Người dùng nhấn vào nút "Show more". \\
        & 6.2. Người dùng chọn lựa chọn "Thời lượng tính bằng phút" tại hàng "Thời lượng". \\
        & 6.3. Người dùng nhập thông tin vào ô dưới "Thời lượng tính bằng phút". \\
        & Quay lại bước 6 trên Normal flow. \\
        & \\
        & A5. Tại bước 6: \\
        & 6.1. Người dùng nhấn vào nút "Show more". \\
        & 6.2. Người dùng chọn vào ô "Lặp lại sự kiện này". \\
        & 6.3. Người dùng nhập thông tin vào ô "Lặp lại hàng tuần, tạo ra tất cả". \\
        & Quay lại bước 6 trên Normal flow. \\
        & \\
        & A6. Tại bước 5: \\
        & 5.1. Người dùng chọn "Mục" tại hàng "Loại sự kiện". \\
        & 5.2. Người dùng chọn các mục cần thiết tại hàng "Mục". \\
        & Quay lại bước 6 trên Normal flow. \\
        \hline
        Exception flows & \\
        & E1. Tại bước 3: \\
        & 3.1. Người dùng không nhập vào ô "Tên sự kiện". \\
        & 3.2. Người dùng nhấn nút "Lưu". \\
        & 3.3. Hệ thống báo lỗi "Bắt buộc". \\
        & Quay lại bước 3 trên Normal flow. \\
        & \\
        & E2. Tại bước 6: \\
        & 6.1. Người dùng nhấn vào nút "Show more". \\
        & 6.2. Người dùng chọn lựa chọn "Thời lượng tính bằng phút" tại hàng "Thời lượng". \\
        & 6.3. Người dùng nhập thông tin không phải số nguyên dương vào ô dưới "Thời lượng tính bằng phút". \\
        & 6.4. Người dùng nhấn nút "Lưu". \\
        & 6.5. Hệ thống báo lỗi "Khoảng thời gian tính bằng phút mà bạn vừa nhập không có hiệu lực. Vui lòng nhập một khoảng thời gian tính bằng phút lớn hơn 0 hoặc không chọn thời gian.". \\
        & Quay lại bước 6.2 trên Alternative flow A4. \\
        & \\
        & E3. Tại bước 6: \\
        & 6.1. Người dùng nhấn vào nút "Show more". \\
        & 6.2. Người dùng chọn lựa chọn "Tới" tại hàng "Thời lượng". \\
        & 6.3. Người dùng chọn thông tin ngày, tháng, năm, giờ, phút kết thúc của sự kiện diễn ra trước thời điểm bắt đầu. \\
        & 6.4. Người dùng nhấn vào nút "Lưu". \\
        & 6.5. Hệ thống báo lỗi "Khoảng thời gian ngày và giờ bạn chọn diễn ra trước khi sự kiện bắt đầu. Vui lòng điều chỉnh trước khi tiếp tục.". \\
        & Quay lại bước 6.3 trên Alternative flow A3. \\
        & \\
        E4. Tại bước 5: \\
        & 5.1. Người dùng chọn "Mục" tại hàng "Loại sự kiện". \\
        & 5.2. Người dùng xóa hết các mục tại hàng "Mục". \\
        & 5.3. Người dùng nhấn vào nút "Lưu". \\
        & 5.4. Hệ thống báo lỗi "Hãy chọn một chuyên mục". \\
        & Quay lại bước 5.2 trên Alternative flow A6. \\
        \hline
    \end{tabular}
    \caption{Use case: Create event}
    \label{Use case: Create event}
\end{table}
\subsubsection{Activity diagram}
\subsubsection{Sử dụng phương pháp Boundary value analysis để kiểm thử}
\subsubsection{Sử dụng phương pháp Equivalence class partitioning}
\subsubsection{Sử dụng phương pháp Decision table để kiểm thử}
\subsubsection{Sử dụng phương pháp Use-case testing để kiểm thử}

\subsection{Chức năng Đổi mật khẩu (Change password)}
\subsubsection{Use case}
\begin{table}[H]
    \centering
    \begin{tabular}{|l|p{11cm}|}
        \hline
        Category & Description \\
        \hline
        Use case name & Change password \\
        \hline
        Actor & Người dùng \\
        \hline
        Assumption & Người dùng ở trang \url{https://sandbox.moodledemo.net/}. Ngôn ngữ trang web được chỉnh tiếng Việt. Người dùng đăng nhập thành công dưới bất kì tài khoản nào. Người dùng đang ở trang "Trang chủ". \\
        \hline
        Normal flow & 1. Người dùng nhấn vào nút mũi tên góc phải trên ngay bên cạnh hình đại diện. \\
        & 2. Người dùng nhấn vào "Tùy chọn". \\
        & 3. Người dùng nhấn vào "Đổi mật khẩu". \\
        & 4. Người dùng nhập đúng mật khẩu hiện tại vào ô "Mật khẩu hiện hành".\\
        & 5. Người dùng nhập mật khẩu mới có thể bao gồm cả chữ, số, kí tự vào ô "Mật khẩu mới".\\
        & 6. Người dùng nhập chính xác thông tin đã nhập ở ô "Mật khẩu mới" vào ô "Mật khẩu mới (lại)".\\
        & 7. Người dùng nhấn vào nút "Lưu những thay đổi". \\
        \hline
        Alternative flows & Không có. \\
        \hline
        Exception flows & E1. Tại bước 4: \\
        & 4.1. Người dùng không nhập thông tin ở nội dung "Mật khẩu hiện hành". \\ 
        & Thực hiện tiếp tục bước 5. Sau khi thực hiện bước 7, hệ thống sẽ hiển thị lỗi "Bắt buộc". Quay lại bước 4 trên Normal flow. \\
        & \\
        & E2. Tại bước 5: \\
        & 5.1. Người dùng không nhập thông tin ở nội dung "Mật khẩu mới". \\
        & Thực hiện tiếp tục bước 6. Sau khi thực hiện bước 7, hệ thống sẽ hiển thị lỗi "Bắt buộc". Quay lại bước 5 trên Normal flow. \\
        & \\
        & E3. Tại bước 6: \\
        & 6.1. Người dùng không nhập thông tin ở nội dung "Mật khẩu mới (lại)". \\
        & Thực hiện tiếp tục bước 7. Sau khi thực hiện bước 7, hệ thống sẽ hiển thị lỗi "Bắt buộc". Quay lại bước 6 trên Normal flow. \\
        & \\
        & E4. Tại bước 4: \\
        & 4.1. Người dùng nhập nhập thông tin tại ô "Mật khẩu hiện hành" chưa chính xác. \\
        & Thực hiện tiếp tục bước 5. Sau khi thực hiện bước 7, hệ thống sẽ hiển thị lỗi "Đăng nhập sai, xin vui lòng thử lại". Quay lại bước 4 trên Normal flow. \\
        & \\
        & E5. Tại bước 6 : \\
        & 6.1. Người dùng nhập nhập thông tin tại ô "Mật khẩu mới (lại)" chưa khớp với nội dung đã nhập tại ô "Mật khẩu mới". \\
        & Thực hiện tiếp tục bước 7. Sau khi thực hiện bước 7, hệ thống sẽ hiển thị lỗi "Các mật khẩu không trùng khớp". Quay lại bước 5 trên Normal flow. \\
        \hline
    \end{tabular}
    \caption{Use case: Change password}
    \label{Use case: Change password}
\end{table}
\subsubsection{Activity diagram}
\subsubsection{Sử dụng phương pháp Boundary value analysis để kiểm thử}
\subsubsection{Sử dụng phương pháp Equivalence class partitioning}
\subsubsection{Sử dụng phương pháp Decision table để kiểm thử}
\subsubsection{Sử dụng phương pháp Use-case testing để kiểm thử}

\section{Tổng kết}
\end{document}
